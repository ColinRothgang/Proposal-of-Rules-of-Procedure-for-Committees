\subsection[Meeting]{Rules regarding the weekly internal meetings of the committees}\label{provision:internal meetings}
\subsubsection{Meeting schedule and announcements}
\begin{enumerate}[a)]
	\item At the beginning of the semester the chair of the Committee, decides on a regular meeting time and place for the weekly internal meetings of the committee\\
	\item As soon as the regular meeting place and time is decided it is to be announced to the USB, at least it is to be documented on the central file-sharing system of the USG (the Teamwork spaces), in the following simply referred to as the Teamwork space of the Committee
	\item In exceptional cases the chair of the committee may decide to move a weekly meeting either in time or in place \begin{itemize}
		\item[--] In this case the change needs to be announced by email to all committee members and additionally a note must be published on the committees teamwork page at least 24 hours prior to both the original time slot and the new time slot
	\end{itemize}
\end{enumerate}
\subsubsection[Agendas and Minutes]{Agendas and Minutes}
\begin{enumerate}[a)]
	\item Agenda points are submitted to the chair or any other member made responsible for this task, no less than 24 hours prior to the meeting. The agenda is to be distributed amongst the members via any official channel of internal communication, additionally it is to be published on the Teamwork page of the Committee no less than 24 hours prior to the meeting 
	\item In important and urgent cases the committee may decide to add additional ad-hoc agenda points in the zero-th Agenda point of every meeting
	\item Each committee may either make a member responsible at each meeting for taking the minutes or designate this task to a \emph{Secretary} of the committee. The chair of the committee is responsible of ensuring this task is distributed and is otherwise to be hold responsible themself
	\item The minutes of each meeting include the date, time and place of the meeting, the members present during the meeting as well as a summary of all important points discussed during the meeting (unless it is decided in the meeting that a point should not be included in the minutes)
	\item The minutes are to be published at least on the Teamwork page of the Committee and no later than 24 hours after the meeting has concluded
\end{enumerate}
\subsubsection[Meeting Procedures]{Procedures for Internal Meetings}
Here section 2-9 of article XI of the USG constitution should be applicable (although literally applicable it remains unclear to the author, whether they are intended to apply also to Committee meetings, however in the authors opinion they should). Additionally, the following rules of procedure are amended:
\subsubsection{Chairing of the Meeting}
Meetings shall be chaired by the Committee chair (USG constitution PART B, article IV), which opens the meeting and guides the discussion through the Agenda points. The chair may decide to yield the chairing right to another attendee to introduce a proposal, hold a speech or moderate a discussion on a topic.
\paragraph{Adherence to agenda points}
The chair of the Committee shall make sure that the discussion follows the agenda points. If the discussion during a meeting is no longer related to the agenda points (including ad-hoc points), the chair estimates, whether the discussion is otherwise beneficial and otherwise cuts it. 
The chair also ensures that each agenda point is covered in a meeting, otherwise the agenda point is to be either rejected or postponed. The chair can reject/postpone any agenda point, however this decision can be overruled by an absolute majority of Committee members. When an agenda point is postponed, it must be stated clearly until when it is postponed. Whenever an agenda point is rejected or postponed this and the date until it is postponed must by documented in the minutes of the meeting. By default any postponed agenda point will be an agenda point of the meeting to which it was postponed (and can possibly be postponed again). 


\paragraph{Decision making}
As specified in article IV, section I of the USG constitution decisions are made by a simple majority vote amongst the committee members. As in article XI, section 3 there is a 50\% quorum and votes may either be public or secret. By default they are public (then the names of all votes are to be named in the minutes), they are secret if decided so by a simple (secret) majority vote.\protect\footnote{To be discussed: Should certain votes by secret by default?} In case of a tie the decision is made by the committee chair.

\subsection[Non-internal Meetings]{USG internal Meetings and Meetings with the Jacobs University Leadership and/or Administration}
\subsubsection{USG internal meetings}
  In meetings between USG Committees the section about internal Committee meetings apply, however the chairs of the involved Committees share the sharing rights and duties during the meeting.
  If the USG Parliament meets a Committee, then the meeting is chaired by the chair of the Parliament (i.e. the Presidents office)\protect\footnote{According to the provision \ref{provision:overruling}.}, but otherwise the section about internal Committee meetings apply.
\subsubsection{Meetings with Jacobs University Leadership and/or Administration}
  If a Committee (or the entire executive unit) meets Jacobs University Leadership and/or Administration and is also making the agenda, the rules regarding Agendas and Minutes in provision \ref{provision:internal meetings} for internal meetings except for point b) apply, otherwise only the rules regarding the minutes are applicable. 

\subsection[Work groups]{Committee work groups}
A Committee work group is a group of Committee members and possibly further USB members working on a specific task assigned by the Committee. The work of a work group should whenever feasible follow a fixed time schedule to be decided on when the task is assigned and the work group is formed. The task for the work group is a task for the members of the work group in the sense of article IV, section 3 of the USG constitution. 
A Committee work group is responsible to the respective Committee and shall regularly update the Committee on its progress. %Whenever requested by the Committee chair or  %simple majority (secret) vote of Committee members (the ones involved in the work group have a conflict of interest and must abstain) with a quorum of one third of Committee members present
%at least two committee members (not members of the work group) it shall provide a report to the committee.

\subsection{Official media of communication}
The following applies only to communication of Committee members due to provision \ref{provision:overruling}.

An official medium of communication needs to be a channel that can be used conveniently by all member of the committee and which allows communication also to multiple members at the same time. Furthermore, it should be possible to conveniently check out past messages over a long period of time. At this stage e-mail is the only official medium of communication besides face-to-face communication. 
Other media of communication may be used additionally for informal communication, however it can not replace an official medium and whenever USG members need to be informed, the communication must be via an official medium. No one can legitimately claim to have informed an USG member unless an official medium of communication was used. 

Whenever communication to the entire undergraduate student body is concerned the information should be published at least on the central file-sharing system of the USG (the teamwork pages).

\subsection{Overruling}\label{provision:overruling}
The provisions of these bylaws shall not violate in any way existing German, European or International law, it should not violate the Academic constitution of Jacobs University nor the USG constitution whenever applicable. Furthermore, they shall not affect the Regulations of any other constituency of the USG besides the Committee's. If such provisions exist, they shall be immediately revoked and/or amended.

Regarding USG's Rules of Procedure, in case of conflicts (in particular meeting procedures across units or Committees of the USG) the Parliaments Rules of Procedure overrule the ones of the executive unit (i.e. these Rules of Procedure) and the Rules of Procedure of the executive unit overrule the Rules of Procedure of individual Committees. 

\subsection{Enforcement of the Rules of Procedure}
The chair of each Committee is responsible for enforcing the Rules of Procedure. Failure at fulfilling this task constitutes a reason for impeachment (as described in article IV, section 3 and article XI, section 9 of the USG constitution).